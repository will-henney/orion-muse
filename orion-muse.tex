% Created 2015-10-15 Thu 21:31
\documentclass[11pt]{article}
\usepackage[utf8]{inputenc}
\usepackage[T1]{fontenc}
\usepackage{fixltx2e}
\usepackage{graphicx}
\usepackage{grffile}
\usepackage{longtable}
\usepackage{wrapfig}
\usepackage{rotating}
\usepackage[normalem]{ulem}
\usepackage{amsmath}
\usepackage{textcomp}
\usepackage{amssymb}
\usepackage{capt-of}
\usepackage{hyperref}
\author{William Henney}
\date{\today}
\title{orion-muse}
\hypersetup{
 pdfauthor={William Henney},
 pdftitle={orion-muse},
 pdfkeywords={},
 pdfsubject={},
 pdfcreator={Emacs 24.5.7 (Org mode 8.3.2)}, 
 pdflang={English}}
\begin{document}

\maketitle
\tableofcontents

\section{{\bfseries\sffamily TODO} Important things to follow up}
\label{sec:orgheadline9}
\subsection{[Fe III] 5270}
\label{sec:orgheadline1}
\begin{itemize}
\item This shows lots of wonderful structure in the jet source regions
\begin{itemize}
\item HH529 and counterjet?
\end{itemize}
\item Also strong in NW extension of HH202
\item Shows jet that can be maybe linked with HH203/204
\item Our WFC3 469N filter shows a small field of this around Orion S
\item MUSE spectrum starts at 4595, so we should also have [Fe III] 4702, 4658, which are of similar brightness to 5270, plus a host of weaker ones - see the Manuel notes.
\begin{itemize}
\item Yes, they are seen - 4658 is the best
\end{itemize}
\end{itemize}
\subsection{O II complex at 4650}
\label{sec:orgheadline2}
\begin{itemize}
\item 4649 and 4651 are severely blended
\begin{itemize}
\item Disentangling these is vital for getting an O II density diagnostic
\end{itemize}
\item We have to sum over a large area to get enough s/n to fit for all the O II components
\item We could maybe use the mean wavelength of the 4649+51 blend as a proxy for the 4649/51 ratio, but we would have to correct for the kinematics, using perhaps [O III] or, better yet, an unblended line of the same multiplet
\begin{itemize}
\item But the only one is  4676 and that is too weak
\item We can't really use 4639+42 because that is blended with N II 4643 and N III 4641
\end{itemize}
\end{itemize}
\subsection{[C I] 8727.13}
\label{sec:orgheadline3}
\begin{itemize}
\item This has a different morphology than anything else!
\item Redshifted filament pointing down SSW from Orion S
\end{itemize}
\subsection{[Ar III] 7135.78}
\label{sec:orgheadline4}
\begin{itemize}
\item This is excellent for velocity mapping
\item The velocity resolution is better at longer wavelengths
\item You can see blue-shifted and red-shifted flows easily
\end{itemize}
\subsection{Weaker lines that might be interesting}
\label{sec:orgheadline5}
\begin{itemize}
\item Ne I 8892.22 - similar to O I
\item Ca I] 9052.16 but 9095.09 is missing so maybe something else
\item\relax [O I] 5577.31 need to remove sky line but then there are some interesting little spots like HH 201
\item 5906 very weak line - I had classified it as Si I 5906.22, 5906.15, 5906.418, 5906.92 but this seems very unlikely since it is not seen in any of the low-ionization parts, only near the Trapezium, most strongly in the SW compact bar
\end{itemize}
\subsection{Using the absorption lines}
\label{sec:orgheadline6}
\begin{itemize}
\item The trouble here is that many of the best He II lines are bluer than the spectral range
\item He II 4686 works well
\begin{itemize}
\item Deepest in th2A
\item One problem is that it is phase-dependent in th1C
\end{itemize}
\item He II 5411 has some potential, but is contaminated by [Fe III]
\item O II 4650 is seen in absorbtion right on th2A, but in the nebula it is swamped by the ORL emission lines
\item C IV 5801.35, 5811.97 are clearly seen in th1C spectrum and much weaker in th2A, absent in other trapezium stars
\begin{itemize}
\item Unfortunately, they are very weak in the nebula
\item Requires integration over 15x15 arcsec box to have much s/n
\end{itemize}
\item N III 6633.9 is very interesting
\begin{itemize}
\item Has absorption depth of 0.08 in Orion S region
\item But not seen in any OB stars
\item Seen in th1E, which is G2V spectral type
\begin{itemize}
\item (which Olivares et al 2013) say is not bound to Trapezium
\item And is also eclipsing binary (Morales-Calderón et al 2012)
\item But absorption depth there is only 0.05
\end{itemize}
\item Conclusion must be that we are seeing scattered light from an embedded yellow supergiant that is leaking out.
\item Some T Tauri stars show a Cr I line at 6630 but that has a very low EW \textasciitilde{}0.01 A (Apenzeller et al 1986, Table III)
\item There are also lines around 6480 and 6490 in the nebular scattered light
\begin{itemize}
\item Some stars show a strongish line around 6495
\end{itemize}
\end{itemize}
\item Not an absorption line, but also possibly stellar
\begin{itemize}
\item Wide (50 Angstrom) bup seen around 8600 Angstrom
\end{itemize}
\end{itemize}

\subsection{{\bfseries\sffamily DONE} Looking at the strange broad NIR emission bump}
\label{sec:orgheadline7}
\begin{itemize}
\item Take the difference or ratio between 8570 A and 8552 A
\item Results are very disappointing
\begin{itemize}
\item We see a vague form of the nebula in the ratio image
\item It looks similar to the continuum image, but not exactly the same
\item It doesn't look much like scattered continuum
\item And there is this instrumental tartan pattern superimposed on it
\item And it is also very noisy
\item Certainly not worth bothering with
\item Ratio image is even worse
\end{itemize}
\end{itemize}
\begin{verbatim}
from astropy.io import fits
import numpy as np

cubehdu, = fits.open('muse-hr-data-wavsec6.fits')
continuum = np.nanmean(cubehdu.data[441:446], axis=0)
bump = np.nanmean(cubehdu.data[455:471], axis=0)
fits.PrimaryHDU(header=cubehdu.header, data=bump-continuum).writeto('bump8600-diff.fits', clobber=True)
fits.PrimaryHDU(header=cubehdu.header, data=bump/continuum).writeto('bump8600-ratio.fits', clobber=True)
\end{verbatim}

\subsection{Longer wavelength lines}
\label{sec:orgheadline8}
\begin{center}
\begin{tabular}{rlrl}
7001.92 & O I & 3 & \\
7002.23 & O I & 3 & blend\\
7065.28 & He I & 2 & \\
7135.78 & [Ar III] & 1 & super strong\\
7155.14 & [Fe II] & 4 & \\
7231.34 & C II & 3 & \\
7236.42 & C II & 3 & \\
7254.15 & O I & 3 & Also 7254.45, 7254.53\\
7281.35 & He I & 3 & \\
7290.3 & ? & 4 & possibly [Fe II]\\
7318.39 & [O II] & 1 & Also 7319.99\\
7329.66 & [O II] & 1 & Also 7330.73\\
7377.83 & [Ni II] & 4 & \\
7411.61 & [Ni II] & 5 & \\
7442.30 & N I & 5 & \\
7452.54 & [Fe II] & 4 & \\
7468.31 & N I & 4 & \\
\hline
7751.10 & [Ar III] & 1 & \\
7816.13 & He I & 4 & \\
7890.07 & Ca I] & 4 & \\
7900 & Sky & 4 & Lots of sky lines\\
8000 & Sky & 4 & in this spectral\\
8100 & Sky & 4 & range\\
\hline
8189 & Fe I? & 4 & ID uncertain\\
8200.36 & N I? & 5 & very weak\\
8210.72 & N I & 5 & \\
8216.34 & N I & 4 & \\
8223.14 & N I & 4 & Strongest component\\
8243 & ? & 4 & O I? or Fe II?\\
8240+ & H I & 4 & Lots of Paschen lines\\
8437.96 & H I & 3 & Pa 18\\
8446.36 & O I & 2 & And 8444.25, 8444.76--\\
8467.25 & H I & 2 & Pa 17\\
8502.48 & H I & 2 & Pa 16\\
8545.38 & H I & 2 & Pa 15\\
8578.69 & [Cl II] & 3 & \\
8598.39 & H I & 2 & Pa 14\\
8600 & Bump? & 4 & Maybe scattered stellar\\
8616.95 & [Fe II] & 3 & \\
8665.02 & H I & 2 & Pa 13\\
8680.28 & N I & 4 & Strongest component\\
8683.40 & N I & 4 & \\
8686.15 & N I & 4 & \\
8703.25 & N I & 4 & \\
8711.70 & N I & 4 & \\
8718.83 & N I & 5 & very weak\\
8727.13 & [C I] & 4 & Different!\\
8733.43 & He I & 5 & very weak\\
8750.47 & H I & 2 & \\
\hline
8862.79 & H I & 2 & \\
8892.22 & Ne I & 4 & \\
8996.99 & He I & 5 & \\
9014.91 & H I & 2 & Pa 10\\
9036 & ? & 5 & very low ionization\\
9052.16 & Ca I] & 5 & \\
9068.90 & [S III] & 1 & \\
9095.09 & Ca I] & 5 & \\
9123.60 & [Cl II] & 4 & \\
9204.17 & O II? & 5 & but looks low ionization\\
9210.28 & He I & 4 & \\
9229.01 & H I & 2 & Pa 9\\
\end{tabular}
\end{center}
\section{Extracting lines for comparison with WFC3}
\label{sec:orgheadline45}
\begin{itemize}
\item The interesting sections are wavsec0 to wavsec3
\end{itemize}

\subsection{Extract subsets of the full cubes}
\label{sec:orgheadline33}

\subsubsection{Transposed cubes}
\label{sec:orgheadline15}
\begin{itemize}
\item These are taken from the section cubes that I made \hyperref[sec:orgheadline10]{down here}
\begin{itemize}
\item Origina axis order is RA, Dec, Wav
\end{itemize}
\item They will look like longslit spectra in DS9 - at least that is the idea
\end{itemize}

\begin{enumerate}
\item {\bfseries\sffamily TODO} Stack of horizontal slits: Wav, RA, Dec
\label{sec:orgheadline11}
\begin{itemize}
\item 1 2 3 -> 3 1 2
\end{itemize}
\begin{verbatim}
MDIR=~/Source/Montage/bin
$MDIR/mTranspose muse-hr-data-wavsec0.fits muse-hr-hslit-stack-wavsec0.fits 3 1 2
\end{verbatim}

\begin{itemize}
\item Note that this takes a long time to run - even longer on my laptop
\item Best to do it in a terminal rather than with babel
\item They are too big (17GB each), so I have moved tham to a \texttt{BigFiles/} folder, which I am not synching with my laptop, only with hypatia
\end{itemize}

\item {\bfseries\sffamily TODO} Stack of vertical slits: Wav, Dec, RA
\label{sec:orgheadline12}
\begin{itemize}
\item 1 2 3 -> 3 2 1
\end{itemize}

\item Bin them up to make them easier to use
\label{sec:orgheadline14}

\begin{enumerate}
\item 1 arcsec pixels: 5x5 binning in RA and Dec
\label{sec:orgheadline13}
\begin{itemize}
\item First, work directly on the original cube sections
\item FITS order is RA, Dec, Wav
\item Python order is Wav, Dec, RA
\item Original NY, NX is 1476, 1766
\begin{itemize}
\item If we chop off just 1 pixel on each axis, then we are divisible by 5
\end{itemize}
\item Work on one image at time to simplify things and reduce memory footprint
\end{itemize}
\begin{verbatim}
from __future__ import print_function
import sys
import numpy as np
from astropy.io import fits

def rebin_xyimage(im, mx=5, my=5):
    ny, nx = im.shape
    # Shape of new rebinned array
    nny, nnx = ny//my, nx//mx
    # Shave a bit off original array so shape is multiple of m
    # And then concertina each axis to be 4-dimensional
    im4d = im[:nny*my, :nnx*mx].reshape((my, nny, mx, mmx))
    # Average along the mx, my axes
    return np.nanmean(im4d, axis=(0, 2))


def rebin_hdu(hdu, m=(5, 5)):
    mx, my = m                  # FITS axis order
    nv, ny, nx = hdu.data.shape  # Python axis order
    nny, nnx = ny//my, nx//mx
    newdata = np.empty((nv, nny, nnx))
    for k in range(nv):
        print('Rebinning plane {}/{}'.format(k+1, nv))
        newdata[k] = rebin_xyimage(hdu.data[k], mx, my)
    newhdu = fits.PrimaryHDU(header=hdu.header, data=newdata)
    # New pixel deltas are bigger
    newhdu.header['CDELT1'] *= mx
    newhdu.header['CDELT2'] *= my
    # pix=0.5 is left edge of first pixel
    newhdu.header['CRPIX1'] = 0.5 + (newhdu.header['CRPIX1'] - 0.5)/mx
    newhdu.header['CRPIX2'] = 0.5 + (newhdu.header['CRPIX2'] - 0.5)/my

    return newhdu

if __name__ == '__main__':
    try:
        infilename = sys.argv[1]
        m = sys.argv[2:4]
    except IndexError:
        print('Usage:', sys.argv[0], 'FITSFILE BINX BINY')

    hdu = fits.open(fn)['DATA']
    suffix = '-rebin{:02d}x{:02d}'.format(*m)
    outfilename = infilename.replace('.fits', suffix+'.fits')
    rebin_hdu(hdu, m).writeto(outfilename)
\end{verbatim}

\begin{itemize}
\item These should all be run in an interactive shell
\item First, test it with a small file
\end{itemize}
\begin{verbatim}
python rebin_datacube.py
\end{verbatim}
\end{enumerate}
\end{enumerate}

\subsubsection{Spectral windows for each WFC3 filter}
\label{sec:orgheadline32}
\begin{itemize}
\item In principal the calibration can be done with just integrating the spectrum over the filter T reponse.
\item But we really need to fit Gaussians to the lines
\item Note that pysynphot requires Python 2.7
\item Also requires PYSYN\_CDBS environment variable to be set
\begin{itemize}
\item On linux server
\end{itemize}
\end{itemize}
\begin{verbatim}
export PYSYN_CDBS=/fs/nil/other0/will/CDBS
\end{verbatim}


\begin{enumerate}
\item List of HST filters to use
\label{sec:orgheadline16}
Export this table to \url{all-filters-input.tab} with \texttt{C-c t e} after any modification. 

\begin{center}
\label{tab:orgtable1}

\begin{tabular}{ll}
Instrument & Filter\\
\hline
wfc3 & f469n\\
wfc3 & f487n\\
wfc3 & f502n\\
wfc3 & f547m\\
wfc3 & fq575n\\
wfc3 & f656n\\
wfc3 & f658n\\
wfc3 & fq672n\\
wfc3 & f673n\\
wfc3 & fq674n\\
\hline
wfpc2 & f502n\\
wfpc2 & f547m\\
wfpc2 & f631n\\
wfpc2 & f656n\\
wfpc2 & f658n\\
wfpc2 & f673n\\
\hline
acs & f658n\\
acs & f660n\\
acs & f435w\\
acs & f555w\\
acs & f775w\\
acs & f850lp\\
\end{tabular}
\end{center}


Send all tables to linux server
\begin{verbatim}
rsync -aPq *.tab nil:/fs/nil/other0/will/orion-muse
\end{verbatim}

\begin{verbatim}
def bp_fullname(instrument, filter_):
    if instrument.lower() == 'wfc3':
        return 'wfc3,uvis1,'+filter_.lower()
    elif instrument.lower() == 'acs':
        return 'acs,wfc1,'+filter_.lower()
    elif instrument.lower() == 'wfpc2':
        return 'wfpc2,'+filter_.lower()
    else:
        raise NotImplementedError('Unknown instrument: ' + instrument)
\end{verbatim}


\item {\bfseries\sffamily DONE} [1/1] Print out the mean wavelength and rectangular width of each filter
\label{sec:orgheadline17}
\begin{verbatim}
import pysynphot
from astropy.table import Table
def bp_fullname(instrument, filter_):
    if instrument.lower() == 'wfc3':
        return 'wfc3,uvis1,'+filter_.lower()
    elif instrument.lower() == 'acs':
        return 'acs,wfc1,'+filter_.lower()
    elif instrument.lower() == 'wfpc2':
        return 'wfpc2,'+filter_.lower()
    else:
        raise NotImplementedError('Unknown instrument: ' + instrument)
float_fmt = '{:.2f}'
intab = Table.read('all-filters-input.tab', format='ascii.tab')
outtab = [['Filter', 'Wav0', 'dWav'], None]
for row in intab:
    fn = bp_fullname(row['Instrument'], row['Filter'])
    bp = pysynphot.ObsBandpass(fn)
    outtab.append([fn, float_fmt.format(bp.avgwave()), float_fmt.format(bp.rectwidth())])
\end{verbatim}

\begin{itemize}
\item $\boxtimes$ Test that this works on linux server
\end{itemize}



\item Decomposing the components that go into the throughput curve
\label{sec:orgheadline19}

This is done in \url{wfc3-throughput-components.py}

Use the STScI python install for a change
\begin{verbatim}
source ~/.bash_profile
ur_setup
export PYSYN_CDBS=/Users/will/Dropbox/CDBS
python wfc3-throughput-components.py
\end{verbatim}

This is a plot of all the components that multiply together to make the filter throughput:
\begin{description}
\item[{hst\_ota}] Optical Telescope Assembly.  I think this is the primary mirror efficiency, accounting for fraction of circular area that is obscured by secondary. Roughly constant at about 0.65
\item[{wfc3\_uvis\_ccd1}] Efficiency of CCD, roughly constant at \textasciitilde{} 0.87
\begin{itemize}
\item Note that CCD2 is extremely similar.  The difference is less than 0.5\%
\end{itemize}
\item[{wfc3\_uvis\_owin}] Outer window transmission, roughly 0.95
\item[{wfc3\_uvis\_cor}] Correction based on white dwarf photometry. Roughly 1.18 but falling to red.
\item[{wfc3\_uvis\_mir1}] Internal camera mirror efficiency, roughly 0.9
\item[{wfc3\_uvis\_mir2}] Another mirror efficiency, roughly 0.9
\item[{wfc3\_uvis\_iwin}] Internal window, roughly 0.95
\item[{wfc3\_pom\_001}] Pick Off Mirror (45 deg mirror that diverts light into instrument), roughly 0.88
\item[{wfc3\_uvis\_f547m}] The filter itself, roughly 0.85
\end{description}

Multiplying them all together gives the total transmission of \texttt{0.364878642843}, which matches what is expected for the total bandpass.


\begin{enumerate}
\item {\bfseries\sffamily DONE} Variation with time
\label{sec:orgheadline18}
\begin{itemize}
\item According to \href{http://ssb.stsci.edu/pysynphot/docs/appendixb.html#pysynphot-appendixb}{Appendix B of the pysynphot docs} we can ask for the filter throughput for a particular MJD using e.g., 'wfc3,uvis1,f658n,mjd\#49486'
\item Today's Julian date is \texttt{57307}
\item Orion S observations were around MJD=55933
\item Plot various dates in \url{wfc3-throughput-evolution.py}
\end{itemize}
\begin{verbatim}
source ~/.bash_profile
ur_setup
export PYSYN_CDBS=/Users/will/Dropbox/CDBS
python wfc3-throughput-evolution.py
\end{verbatim}

Upshot is that there is no discernible difference with time, and also that the Quantum Yield Correction makes no difference at the shortest wavelengths that we are interested in (4700 Angstrom), 

\texttt{2.7.10}
\end{enumerate}

\item {\bfseries\sffamily TODO} Converting surface brightness to predicted counts
\label{sec:orgheadline20}
\begin{itemize}
\item Note that \texttt{bp.primary\_area} is given as 45238.93416, which must be in sq cm.  This is same as \texttt{45238.9342117}
\item The units of the muse data is given as '10**(-20)*erg/s/cm**2/Angstrom'
\begin{itemize}
\item This must be per pixel, I assume
\item Each pixel is 0.2 arcsec square, so this is \texttt{9.40175455274e-13} steradian
\end{itemize}
\item $\square$ We have summed this in wavelength, but we really should have multiplied by lambda first to convert from energy to photon units
\begin{itemize}
\item And while we are at it we can use the C\(_{\text{WFC3}}\) to put it in electron/s
\end{itemize}
\end{itemize}



\item Air to vacuum wavelength conversion
\label{sec:orgheadline21}
\begin{itemize}
\item This depends on refractive index of air, given by the following function
\item To convert air -> vacuum we multiply the wavelengths by the refractive index
\item 
\end{itemize}

\begin{verbatim}
from astropy import units as u
def air_refractive_index(wav):
    """Equation (65) of Greisen et al 2006 for the refractive index of air
at STP.  Input wavelength 'wav' should be in microns or in any
'astropy.units' unit. It does not matter if 'wav' is on air or vacuum
scale

    """
    try:
        # Convert to microns if necessary
        wavm = wav.to(u.micron).value
    except AttributeError:
        # Assume already in microns
        wavm = wav
    return 1.0 + 1e-6*(287.6155 + 1.62887/wavm**2 + 0.01360/wavm**4)
\end{verbatim}
\item {\bfseries\sffamily TODO} [5/5] Process spectral windows for each filter
\label{sec:orgheadline27}
\begin{itemize}
\item This could be the last step that we would have to run on the server
\item If the files are small enough then they can be copied over to the macs
\item Each of the following snippets is run interactively on the server
\end{itemize}
\begin{enumerate}
\item {\bfseries\sffamily DONE} Imports
\label{sec:orgheadline22}
\begin{verbatim}
from astropy.io import fits
from astropy import wcs
from astropy.table import Table
import pysynphot
import numpy as np
\end{verbatim}
\item {\bfseries\sffamily DONE} Read FITS cube
\label{sec:orgheadline23}
\begin{verbatim}
hdulist = fits.open('DATA/DATACUBEFINALuser_20140216T010259_78380e1d.fits')
\end{verbatim}
\item {\bfseries\sffamily DONE} Set up a vacuum wavelength scale
\label{sec:orgheadline24}
\begin{verbatim}
from astropy import units as u
def air_refractive_index(wav):
    """Equation (65) of Greisen et al 2006 for the refractive index of air
at STP.  Input wavelength 'wav' should be in microns or in any
'astropy.units' unit. It does not matter if 'wav' is on air or vacuum
scale

    """
    try:
        # Convert to microns if necessary
        wavm = wav.to(u.micron).value
    except AttributeError:
        # Assume already in microns
        wavm = wav
    return 1.0 + 1e-6*(287.6155 + 1.62887/wavm**2 + 0.01360/wavm**4)

w = wcs.WCS(hdulist['DATA'].header)
NV, NY, NX = hdulist['DATA'].data.shape
# construct array of observed air wavelengths (at image center to be safe)
_, _, wavs = w.all_pix2world([NX/2]*NV, [NY/2]*NV, np.arange(NV), 0) 
# Make dimensional
wavs *= u.m
# Convert to vacuum scale
wavs *= air_refractive_index(wavs)
\end{verbatim}
\item {\bfseries\sffamily DONE} Read in the table of filters
\label{sec:orgheadline25}
\begin{verbatim}
intab = Table.read('all-filters-input.tab', format='ascii.tab')
\end{verbatim}
\item {\bfseries\sffamily DONE} Extract the windows for each filter
\label{sec:orgheadline26}
\begin{verbatim}
def bp_fullname(instrument, filter_):
    if instrument.lower() == 'wfc3':
        return 'wfc3,uvis1,'+filter_.lower()
    elif instrument.lower() == 'acs':
        return 'acs,wfc1,'+filter_.lower()
    elif instrument.lower() == 'wfpc2':
        return 'wfpc2,'+filter_.lower()
    else:
        raise NotImplementedError('Unknown instrument: ' + instrument)
for row in intab:
    bpname = bp_fullname(row['Instrument'], row['Filter'])
    bp = pysynphot.ObsBandpass(bpname)
    # extend a full rectwidth either side of the average wavelength to fit it all in
    wav_window = bp.avgwave() + bp.rectwidth()*np.array([-1, 1])
    # Add in the units (all are in Angstrom I hope)
    assert bp.waveunits.name == 'angstrom'
    wav_window *= u.Angstrom
    # convert to air wavelengths to agree with the WCS
    wav_window /= air_refractive_index(wav_window)
    # Now convert to fractional pixel coordinates
    _, _, [k1, k2] = w.all_world2pix([0, 0], [0, 0], wav_window.to(u.m), 0)
    # smallest slice that covers the window
    wavslice = slice(int(k1), int(k2) + 2)
    # tuple of slices for the 3 cube axes (in numpy array order: V, Y, X)
    cubeslices = [wavslice, slice(None, None), slice(None, None)]

    newhdr = hdulist['DATA'].header.copy()
    newhdr.update(w.slice(cubeslices).to_header())

    # Make a new HDUlist for the windowed spectrum and write it out
    fits.HDUList(
        [fits.PrimaryHDU(header=hdulist[0].header, data=None),
         fits.ImageHDU(header=newhdr, data=hdulist['DATA'].data[cubeslices])
        ]
    ).writeto('muse-hr-window-{}-{}.fits'.format(row['Instrument'], row['Filter']), clobber=True)
\end{verbatim}
\end{enumerate}
\item Cleaning up the window FITS files for DS9
\label{sec:orgheadline28}
For some reason, ds9 does not like the wavelength WCS, so we will try and fix it:
\begin{itemize}
\item Put the physical scales in the CDELTi instead of in the PCi\_j
\item Put it in angstrom instead of m
\item That's it to start with
\end{itemize}
\begin{verbatim}
import sys
from astropy.io import fits
def clean_up_wav_wcs(filename):
    hdulist = fits.open(filename, mode='update')
    for hdu in hdulist:
        if hdu.header.get('CUNIT3') == 'm':
            # Change to Angstrom
            hdu.header['PC3_3'] *= 1e10
            hdu.header['CRVAL3'] *= 1e10
            hdu.header['CUNIT3'] = 'Angstrom'
            # And move scales to CDELT
            for i in '123':
                CDELTi = 'CDELT'+i
                # Sanity check
                assert hdu.header.get(CDELTi) == 1.0
                PCi_j = 'PC{0}_{0}'.format(i)
                hdu.header[CDELTi], hdu.header[PCi_j] = hdu.header[PCi_j], hdu.header[CDELTi] 
    hdulist.flush()


if __name__ == '__main__':
    try:
        fn = sys.argv[1]
        clean_up_wav_wcs(fn)
    except IndexError:
        print('Usage:', sys.argv[0], 'FITSFILE')
\end{verbatim}
Export with \texttt{C-u C-c C-v C-t}

Test it on the WFC3 f656n file

\begin{verbatim}
python clean_up_wav_wcs.py muse-hr-window-wfc3-fq674n.fits
\end{verbatim}

That seemed to work

\begin{verbatim}
python clean_up_wav_wcs.py muse-hr-window-wfc3-f487n.fits
\end{verbatim}

\item {\bfseries\sffamily DONE} Convert from erg/cm2/s/Angstrom to electron/s
\label{sec:orgheadline29}
\begin{itemize}
\item The fundamental equation is \(R_{}_j = C_{WFC3 }\int \lambda I_\lambda T_\lambda d\lambda\)
\begin{itemize}
\item Where C\(_{\text{WFC3 }}\)= 0.0840241 if \(\lambda\) is in \AA{}
\end{itemize}
\item So, we need to multiply by lambda when we do the flattening
\item Also, we need to get from MUSE's flux-per-pixel to brightness (per-steradian)
\begin{itemize}
\item This means we divide by the MUSE pixel area of 9.40175455274e-13 sr
\end{itemize}
\item AND we need to multiply by the MUSE bin width in \AA{}
\item Question is, do we apply this normalization to the \texttt{transwin} cubes?
\begin{itemize}
\item Best not, so as to minimze churn of large files on Dropbox
\end{itemize}
\end{itemize}

\begin{verbatim}
from astropy import units as u
WFC3_CONSTANT = 0.0840241
MUSE_FLUX_UNITS = 1e-20 
MUSE_PIXEL_AREA_SR = (0.2*u.arcsec).to(u.radian)**2
\end{verbatim}
\item Fold the spectra through each filter to get simulated images
\label{sec:orgheadline30}
This does not have to be done on the server any more

\begin{verbatim}
from __future__ import print_function
import sys
from astropy.io import fits
from astropy import wcs
from astropy.table import Table
import pysynphot
import numpy as np
from astropy import units as u
def air_refractive_index(wav):
    """Equation (65) of Greisen et al 2006 for the refractive index of air
at STP.  Input wavelength 'wav' should be in microns or in any
'astropy.units' unit. It does not matter if 'wav' is on air or vacuum
scale

    """
    try:
        # Convert to microns if necessary
        wavm = wav.to(u.micron).value
    except AttributeError:
        # Assume already in microns
        wavm = wav
    return 1.0 + 1e-6*(287.6155 + 1.62887/wavm**2 + 0.01360/wavm**4)

def bp_fullname(instrument, filter_):
    if instrument.lower() == 'wfc3':
        return 'wfc3,uvis1,'+filter_.lower()
    elif instrument.lower() == 'acs':
        return 'acs,wfc1,'+filter_.lower()
    elif instrument.lower() == 'wfpc2':
        return 'wfpc2,'+filter_.lower()
    else:
        raise NotImplementedError('Unknown instrument: ' + instrument)
from astropy import units as u
WFC3_CONSTANT = 0.0840241
MUSE_FLUX_UNITS = 1e-20 
MUSE_PIXEL_AREA_SR = (0.2*u.arcsec).to(u.radian)**2


def bandpass_flatten(instrument, bpname):
    filename = 'muse-hr-window-{}-{}.fits'.format(instrument, bpname)
    hdulist = fits.open(filename)
    hdu = hdulist['DATA']
    w = wcs.WCS(hdu.header)
    NV, NY, NX = hdu.data.shape
    # construct array of observed air wavelengths (at image center to be safe)
    _, _, wavs = w.all_pix2world([NX/2]*NV, [NY/2]*NV, np.arange(NV), 0) 
    # Make dimensional
    wavs *= u.m
    # Convert to vacuum scale
    wavs *= air_refractive_index(wavs)

    # Get bandpass for filter
    fn = bp_fullname(instrument, bpname)
    bp = pysynphot.ObsBandpass(fn)
    # Calculate transmission curve at the observed wavelengths
    T = bp(wavs.to(u.Angstrom).value)
    # Weight by transmission curve and save that
    hdu.data *= T[:, None, None]
    hdulist.writeto(filename.replace('-window-', '-transwin-'), clobber=True)
    # Integrate over wavelength, already weighted by transmission curve. But
    # now need to multiply by wavelength, put in brightness units, and
    # convert to WFC3 electron/s/pixel
    hdu.data *= WFC3_CONSTANT*MUSE_FLUX_UNITS/MUSE_PIXEL_AREA_SR
    hdu.data *= wavs.to(u.Angstrom).value[:, None, None]
    hdu.data = hdu.header['CDELT3']*np.sum(hdu.data, axis=0)
    hdu.header['BUNIT'] = 'electron/s/(0.03962 arcsec)**2'
    hdulist.writeto(filename.replace('-window-', '-image-'), clobber=True)

if __name__ == '__main__':
    try:
        instrument, bpname = sys.argv[1:]
        bandpass_flatten(instrument, bpname)
    except IndexError:
        print('Usage:', sys.argv[0], 'INSTRUMENT FILTER')
\end{verbatim}

New example of use, using STSCI python on laptop
\begin{verbatim}
source ~/.bash_profile
ur_setup
export PYSYN_CDBS=/Users/will/Dropbox/CDBS
python filter-flatten.py wfc3 fq575n
\end{verbatim}


Check the same one using the Anaconda py27 on hypatia, but make sure that we are using the same version of CDBS
\begin{verbatim}
source activate py27
export PYSYN_CDBS=/Users/will/Dropbox/CDBS
python filter-flatten.py wfc3 fq575n
\end{verbatim}
Exactly the same, which is a heartening.  

Now do it for all the filters
\begin{verbatim}
source activate py27
export PYSYN_CDBS=/Users/will/Dropbox/CDBS
FILTERS="f469n f487n f502n f547m fq575n f656n f658n fq672n f673n fq674n"
for f in $FILTERS; do
    echo Flattening $f
    python filter-flatten.py wfc3 $f
done
\end{verbatim}
\item Comparing profiles by eye
\label{sec:orgheadline31}
\begin{itemize}
\item Took some profiles by eye on WFC3 and MUSE images of F547M
\begin{itemize}
\item Cannot compare them in DS9 because the spatial axis is written in pixels, which are different sizes
\end{itemize}
\item \url{muse-f547m-cut.dat}
\item \url{wfc3-f547m-cut.dat}
\end{itemize}
\end{enumerate}
\subsection{{\bfseries\sffamily DONE} [4/4] Compare the real and predicted count-rate images on a common grid}
\label{sec:orgheadline38}
\begin{itemize}
\item We want to put everything on the MUSE pixel grid, since that will make smaller files by a factor of \texttt{25.4818555174}
\item We could use
\begin{enumerate}
\item astrodrizzle
\begin{itemize}
\item \href{file:///Users/will/Dropbox/OrionHST-2012/HST-ACS/acs-ramp-filters.org}{acs-ramp-filters.org}
\end{itemize}
\item montage
\begin{itemize}
\item \url{https://montage-wrapper.readthedocs.org}
\item I had already done that in the t-squared \href{../../Work/RubinWFC3/Tsquared/orion-t2.org}{project}
\item And I hadn't even used the python bindings
\end{itemize}
\end{enumerate}
\end{itemize}
\subsubsection{{\bfseries\sffamily DONE} Check that I have a working Montage installation}
\label{sec:orgheadline34}
\begin{itemize}
\item I already have version 3.3 but version 4 is out
\item Cloning from github into \url{file:///Users/will/Source/Montage/}
\item Compiled with \texttt{make -j8} - that was fast!
\end{itemize}
\subsubsection{{\bfseries\sffamily DONE} Testing out Montage}
\label{sec:orgheadline35}
\begin{verbatim}
PATH=$PATH:~/Source/Montage/bin
err=$(mProjectPP --help)
echo $err
\end{verbatim}
\subsubsection{{\bfseries\sffamily DONE} Script to resample WFC3 image onto MUSE grid}
\label{sec:orgheadline36}
\begin{itemize}
\item Header for MUSE full frame grid: \url{muse-full-frame.hdr}
\item This does not expand the WFC3 image beyond its original borders
\begin{itemize}
\item So we will have to extract a section of the MUSE image for comparison
\item On the other hand, it does maintain the same reference pixel
\begin{itemize}
\item But with different values of CRPIX because the image lower left corner is different
\item So it has the same values of CRVAL
\item This will make it easy to slice the MUSE image
\end{itemize}
\end{itemize}
\item Smoothing needs to be improved
\begin{itemize}
\item No smoothing is too little
\item The \texttt{s120} images that I already have are too much (this was 1.2 arcsec I assume)
\item In the \href{../../Work/RubinWFC3/Tsquared/orion-t2.org}{t2 notes} I calculated FWHM of 4 pixels = 0.8 arcsec
\item Actually 0.7 arcsec was better - this is now done in \href{../../Work/RubinWFC3/Tsquared/orion-t2.org}{the orion-t2.org notes}
\end{itemize}
\end{itemize}
\begin{verbatim}
F=$1
MDIR=~/Source/Montage/bin
TDIR=~/Work/RubinWFC3/Tsquared
$MDIR/mProjectPP -h 0 -X $TDIR/full_${F}-s070.fits wfc3-resample-muse-$F.fits muse-full-frame.hdr
\end{verbatim}

Test with a single image
\begin{verbatim}
time sh wfc3-resample-to-muse.sh F547M 2>&1
\end{verbatim}


Do all of the images
\begin{verbatim}
FILTERS="f469n f487n f502n f547m fq575n f656n f658n fq672n f673n fq674n"
for f in $FILTERS; do
    echo "Resampling $f"
    time sh wfc3-resample-to-muse.sh $f
done
\end{verbatim}
\subsubsection{{\bfseries\sffamily DONE} [2/2] Crop MUSE image to the WFC3 field}
\label{sec:orgheadline37}
\begin{itemize}
\item $\boxtimes$ This is the final step required before we can do things like take ratio maps or calculate 2d histogram images
\item $\boxtimes$ \textit{[2015-10-15 Thu] } Also, write out the integrated spectrum times filter throughput for the cropped region
\end{itemize}
\begin{verbatim}
import sys
import numpy as np
from astropy.io import fits
from astropy.wcs import WCS, WCSSUB_SPECTRAL
import astropy.units as u

def crop_muse_to_wfc3(fid):
    """Cut out a section of the MUSE image to match the WFC3 field"""
    wname = 'wfc3-resample-muse-{}.fits'.format(fid)
    mname = 'muse-hr-image-wfc3-{}.fits'.format(fid)
    whdu = fits.open(wname)[0]
    mhdu = fits.open(mname)['DATA']
    # Also get the spectral data cube multiplied by filter throughput
    shdu = fits.open(mname.replace('-image-', '-transwin-'))['DATA']
    wcs_w = WCS(whdu.header).celestial
    wcs_m = WCS(mhdu.header).celestial
    # Check that the two images have the same reference point in RA, DEC
    assert np.all(wcs_w.wcs.crval == wcs_m.wcs.crval)
    # And that the pixel scales are the same
    assert np.all(wcs_w.wcs.cdelt == wcs_m.wcs.cdelt)
    assert np.all(wcs_w.wcs.pc == wcs_m.wcs.pc)

    # The shapes of the two grids: (nx, ny) in FITS axis order
    shape_w = np.array([whdu.header['NAXIS1'], whdu.header['NAXIS2']])
    shape_m = np.array([mhdu.header['NAXIS1'], mhdu.header['NAXIS2']])

    # The difference in CRPIX values tells us the start indices (i, j)
    # for the crop window on the MUSE grid. Note that this is in
    # zero-based array indices
    start = wcs_m.wcs.crpix - wcs_w.wcs.crpix
    # The stop indices for the crop window 
    stop = start + shape_w

    # Shift 1 pixel to the right to do a coarse alignment correction
    start[0] += 1
    stop[0] += 1

    # Check that these are within bounds of the original MUSE grid
    assert np.all(start >= 0.0)
    assert np.all(stop < shape_m)

    # Crop the MUSE data array to the start:stop indices, remembering
    # that python axis order is backwards with respect to FITS axis
    # order
    mhdu.data = mhdu.data[start[1]:stop[1], start[0]:stop[0]]

    # And copy the WFC3 wcs into the new MUSE header
    mhdu.header.update(wcs_w.to_header())

    # Write out the new cropped MUSE image
    oname = mname.replace('-image-', '-cropimage-')
    mhdu.writeto(oname, clobber=True)

    # Finally, as a bonus, calculate the 1-D average spectrum from the cube
    spec = np.nanmean(shdu.data[:, start[1]:stop[1], start[0]:stop[0]], axis=(-1, -2))

    # Convert from 1e-20 flux-per-pixel to surface brightness units (flux per sr)
    pixel_area_sr = np.product(np.abs(wcs_m.wcs.cdelt))*(u.deg.to(u.radian))**2
    spec *= 1e-20/pixel_area_sr
    # extract only the spectral part of the cube's WCS
    wcs_s = WCS(shdu.header).sub([WCSSUB_SPECTRAL])
    oshdu = fits.PrimaryHDU(header=wcs_s.to_header(), data=spec)
    oshdu.header['BUNIT'] = 'erg/s/cm**2/sr/Angstrom'
    # Fix up the wavelngth units to angstrom
    oshdu.header['CDELT1'] *= 1e10
    oshdu.header['CRVAL1'] *= 1e10
    oshdu.header['CUNIT1'] = 'Angstrom'
    # And record the window from the MUSE full field that was extracted
    oshdu.header['MUSE_X1'] = start[0] + 1, 'Extracted window: start X pixel' 
    oshdu.header['MUSE_X2'] = stop[0] + 1, 'Extracted window: stop X pixel' 
    oshdu.header['MUSE_Y1'] = start[1] + 1, 'Extracted window: start Y pixel' 
    oshdu.header['MUSE_Y2'] = stop[0] + 1, 'Extracted window: stop Y pixel' 
    oshdu.writeto(mname.replace('-image-', '-cropspec1d-'), clobber=True)

    return oname


if __name__ == '__main__':
    try:
        filter_id = sys.argv[1]
    except:
        print('Usage:', sys.argv[0], 'FILTER')

    print(crop_muse_to_wfc3(filter_id))
\end{verbatim}

\begin{verbatim}
python crop_muse.py f547m
\end{verbatim}

\begin{verbatim}
FILTERS="f469n f487n f502n f547m fq575n f656n f658n fq672n f673n fq674n"
for f in $FILTERS; do
    time python crop_muse.py $f
done
\end{verbatim}
\subsection{Plot the cropped 1D spectra}
\label{sec:orgheadline39}
\begin{verbatim}
from __future__ import print_function
import sys
from astropy.io import fits
from astropy.wcs import WCS
from astropy import units as u
from matplotlib import pyplot as plt
import seaborn as sns

def plot_1d_spec_from_fits(fn, ax, fontsize=None):
    """Plots spectrum from filename `fn` onto pre-existing axis `ax`"""
    hdu = fits.open(fn)[0]
    spec = hdu.data/1e-3
    w = WCS(hdu.header)
    nwav = len(spec)
    wavs, = w.all_pix2world(range(nwav), 0)
    wavs *= u.m.to(u.Angstrom)
    #ax.plot(wavs, spec, drawstyle='steps-mid')
    ax.bar(wavs, spec, align='center', linewidth=0)
    ax.set_xlim(wavs.min(), wavs.max())
    ax.set_xlabel('Observed Air Wavelength, Angstrom', fontsize=fontsize)
    ax.set_ylabel('Filter Throughput x Brightness\n 0.001 erg/s/cm^2/sr/Angstrom', fontsize=fontsize)


if __name__ == '__main__':
    try:
        filt = sys.argv[1]
    except IndexError:
        print('Usage:', sys.argv[0], 'FILTER')
    fig, ax = plt.subplots(1, 1)
    fn = 'muse-hr-cropspec1d-wfc3-{}.fits'.format(filt)
    plot_1d_spec_from_fits(fn, ax)
    # ax.set_yscale('log')
    # ax.set_ylim(1e-7, None)
    fig.savefig(sys.argv[0].replace('.py', '-test-{}.pdf'.format(filt)))
\end{verbatim}

\begin{verbatim}
FILTERS="f469n f487n f502n f547m fq575n f656n f658n fq672n f673n fq674n"
for f in $FILTERS; do
    python specplot1d_utils.py $f
done
\end{verbatim}

\subsection{Visualizations of throughput calibration quality}
\label{sec:orgheadline43}
The final count-rate images to be compared are 
\begin{description}
\item[{Smoothed WFC3}] wfc3-resample-muse-FILTER.fits
\item[{Cropped MUSE}] muse-hr-cropimage-wfc3-FILTER.fits
\end{description}
\subsubsection{Ratios of the images}
\label{sec:orgheadline40}
\begin{itemize}
\item This will allow us to see how important misalignment is, and if there are any spatial trends
\end{itemize}

\begin{verbatim}
from astropy.io import fits

filters_ = ["FQ575N", "FQ672N", "FQ674N", "F673N", "F469N",
            "F487N", "F656N", "F658N", "F547M", "F502N"]

def divide_fits_images(name1, name2, outname):
    hdu1 = fits.open(name1)[0]
    hdu2 = fits.open(name2)['DATA']
    fits.PrimaryHDU(header=hdu1.header, data=hdu1.data/hdu2.data).writeto(outname, clobber=True)

if __name__ == '__main__':
    for f in filters_:
        divide_fits_images(
            'wfc3-resample-muse-{}.fits'.format(f),
            'muse-hr-cropimage-wfc3-{}.fits'.format(f),
            'wfc3-over-muse-calib-ratio-{}.fits'.format(f)
        )
\end{verbatim}


\subsubsection{Two-d histogram of WFC3 vs MUSE-predicted count rates}
\label{sec:orgheadline41}

\begin{verbatim}
from __future__ import print_function
import numpy as np
from astropy.io import fits
from astropy.convolution import convolve, Gaussian2DKernel
from matplotlib import pyplot as plt
import seaborn as sns
from specplot1d_utils import plot_1d_spec_from_fits

maxcount = {
    "fq575n": 0.4,
    "fq672n": 0.6,
    "fq674n": 0.75,
    "f673n" : 2.5,
    "f469n" : 0.5,
    "f487n" : 10.0,
    "f656n" : 40.0, 
    "f658n" : 11.0, 
    "f547m" : 7.0, 
    "f502n" : 20.0,
}
GAMMA = 2.0

cmap = sns.light_palette((260, 50, 30), input="husl", as_cmap=True)
# cmap = plt.cm.gray_r

def histogram_calib_images(f, vmax=1.0):
    name1 = 'wfc3-resample-muse-{}.fits'.format(f)
    name2 = 'muse-hr-cropimage-wfc3-{}.fits'.format(f)
    pltname = 'wfc3-vs-muse-calib-{}.pdf'.format(f)
    hdu1 = fits.open(name1)[0]
    hdu2 = fits.open(name2)['DATA']
    hduc = fits.open('wfc3-resample-muse-f547m.fits')[0]
    x, y = hdu2.data, hdu1.data
    xmin, xmax = ymin, ymax = 0.0, vmax
    ew = y/hduc.data
    # mask out silly values
    m = np.isfinite(x) & np.isfinite(y/x) & (np.abs(np.log10(y/x)) < 1.0)
    H, xedges, yedges = np.histogram2d(x[m], y[m], 200,
                                       [[xmin, xmax], [ymin, ymax]],
                                       weights=y[m])
    # Fit a straight line
    mm = m & (x > 0.05*xmax) & (y > 0.05*ymax) & (x < 0.5*xmax) & (y < 0.5*ymax) & (np.abs(np.log10(y/x)) < 0.3)
    # First, linear fit y(x) = m x + c
    y_x_linfit = np.polyfit(x[mm], y[mm], 1, w=y[mm])
    # Second, linear fit x(y) = m y + c
    x_y_linfit = np.polyfit(y[mm], x[mm], 1, w=y[mm])
    # Convert this from x(y) -> y(x)
    # If x = m y + c, then y = (1/m) x - c/m
    y_x_altfit = np.array([1./x_y_linfit[0], -x_y_linfit[1]/x_y_linfit[0]])
    # Take average and spread of these two fits
    y_x_bestfit = 0.5*(y_x_linfit + y_x_altfit)
    y_x_errfit = 0.5*np.abs(y_x_linfit - y_x_altfit)

    pbest =  np.poly1d(y_x_bestfit)

    # H = convolve(H, Gaussian2DKernel(1.0))
    fitcolor = (1.0, 0.5, 0.0)
    fitlabel = "y = ({:.2f} +/- {:.2f}) x + ({:.2f} +/- {:.2f})".format(
        y_x_bestfit[0], y_x_errfit[0], y_x_bestfit[1], y_x_errfit[1])
    fig, ax = plt.subplots(1, 1)
    ax.imshow((H.T)**(1.0/GAMMA), extent=[xmin, xmax, ymin, ymax],
              interpolation='none', aspect='auto', origin='lower', 
              cmap=cmap, alpha=1.0)
    ax.plot([0.0, x[m].max()], [0.0, x[m].max()], '-', alpha=1.0,
            lw=1, c='w', label=None)
    ax.plot([0.0, x[m].max()], pbest([0.0, x[m].max()]), '-', alpha=1.0,
            lw=1, c=fitcolor, label=fitlabel)

    leg = ax.legend(loc='upper left', title='Linear Fit', frameon=True, fancybox=True)
    leg.get_title().set_fontsize('small')
    ax.set_ylabel(
        'Observed WFC3 {} count rate, electron/s/pixel'.format(f.upper()))
    ax.set_xlabel(
        'MUSE-predicted WFC3 {} count rate, electron/s/pixel'.format(f.upper()))
    ax.set_xlim(xmin, xmax)
    ax.set_ylim(ymin, ymax)

    # Now do 1-D histogram of the deviations from the model
    ratio = y/pbest(x)
    if f == 'f547m':
        # Divide into high and low continuum counts
        s = 'counts'
        mlo = (y < y[m].mean()) & m
        mhi = (y >= y[m].mean()) & m
    else:
        # Divide into high and low EW
        s = f.upper() + '/F547M'
        mlo = (ew < np.median(ew[m])) & m
        mhi = (ew >= np.median(ew[m])) & m

    assert mlo.sum() > 0, f

    # inset axis at the top left
    ax2 = fig.add_axes([0.2, 0.55, 0.25, 0.25])
    ax2.hist(ratio[mlo], bins=100, range=(0.5, 1.5),
             normed=True, weights=y[mlo], alpha=0.7, label='Low '+s)
    ax2.hist(ratio[mhi], bins=100, range=(0.5, 1.5),
             normed=True, weights=y[mhi], color='red', alpha=0.3, label='High '+s)
    ax2.set_xlim(0.5, 1.5)
    # leave more space at top
    y1, y2 = ax2.get_ylim()
    y2 *= 1.2
    ax2.set_ylim(y1, y2)
    ax2.legend(loc='upper left', fontsize='xx-small')
    ax2.tick_params(labelleft=False, labelsize='x-small')
    ax2.set_xlabel('(Observed Counts) / (Linear Fit)', fontsize='xx-small')
    ax2.set_ylabel('Weighted PDF Histograms', fontsize='xx-small')
#    ax2.set_title('PDF', fontsize='x-small')

    # inset axis at the bottom right
    ax3 = fig.add_axes([0.6, 0.2, 0.3, 0.3])
    fn = 'muse-hr-cropspec1d-wfc3-{}.fits'.format(f)
    plot_1d_spec_from_fits(fn, ax3, fontsize='xx-small')
    ax3.tick_params(labelsize='xx-small')
    ax3.set_title(f.upper())


    fig.set_size_inches(7, 7)
    fig.savefig(pltname)

    return [pltname, fitlabel]


if __name__ == '__main__':
    for f, vmax in maxcount.items():
        print(histogram_calib_images(f, vmax))
\end{verbatim}


\begin{verbatim}
python histocalib.py
\end{verbatim}


\subsubsection{Summary of calibration results}
\label{sec:orgheadline42}
\begin{itemize}
\item Calibration constant is unity in most cases!
\begin{itemize}
\item Exceptions are
\begin{description}
\item[{F469N}] slope = 0.94
\item[{F673N}] slope = 0.97
\item[{F547M}] intercept = -0.11
\end{description}
\end{itemize}
\item Also no evidence of trend with EW
\end{itemize}


\subsection{Useful scripts}
\label{sec:orgheadline44}
\begin{verbatim}
open -n -a SAOImage\ DS9 --args -title $DS9
sleep 1
xpaset -p $DS9 view buttons no
xpaset -p $DS9 frame delete all
\end{verbatim}

\begin{verbatim}
xpaset -p ds9 frame new
xpaset -p ds9 fits $PWD/muse-hr-data-wavsec3.fits
\end{verbatim}

\section{Exploring the data cubes}
\label{sec:orgheadline53}
\subsection{Original data locations}
\label{sec:orgheadline52}
At CRyA in \texttt{/fs/nil/other0/will/orion-muse/DATA} 
\begin{description}
\item[{LR}] 1.25 Angstrom sampling: DATACUBEFINALuser\_20140216T010259\_cf767044.fits
\item[{HR}] 0.85 Angstrom sampling: DATACUBEFINALuser\_20140216T010259\_78380e1d.fits
\end{description}
\subsubsection{LR cube}
\label{sec:orgheadline48}
\begin{itemize}
\item Dimensions:
\begin{itemize}
\item NV = 3818
\item NY = 1476
\item NX = 1766
\end{itemize}
\item Scales:
\begin{itemize}
\item Spatial: 0.2 arcsec
\item Wavelength: 1.25 Ang
\end{itemize}
\end{itemize}
\begin{enumerate}
\item Reading in the cube
\label{sec:orgheadline46}
\begin{verbatim}
from astropy.io import fits
from astropy import wcs
import numpy as np
hdulist = fits.open('DATA/DATACUBEFINALuser_20140216T010259_cf767044.fits')
cube = hdulist['DATA']
\end{verbatim}
Note that this does not read the full data cube (40 GB) into memory unless we need to do something with it.
\item Extracting the Orion S region
\label{sec:orgheadline47}
\begin{itemize}
\item To start with, we will look at a 300x300 box centered on (1050, 550)
\item This is more or less the quad filter region
\end{itemize}
\begin{verbatim}
subcube = cube.data[:, 400:700, 900:1200]
spec = np.nansum(np.nansum(subcube, axis=-1), axis=-1)
spechdu = fits.PrimaryHDU(header=cube.header, data=spec.reshape((3818, 1, 1)))
spechdu.writeto('subcube-spec.fits')
\end{verbatim}
\begin{itemize}
\item So this gives the summed spectrum of the region
\item Note that I did a reshape on the array so that wavelength is still the 3rd FITS axis.  So that the header WCS keywords don't need changing
\end{itemize}
\begin{verbatim}
rsync -avzP nil:/fs/nil/other0/will/orion-muse/subcube-spec.fits .
\end{verbatim}
\begin{itemize}
\item The spectrum shows up as s single pixel in ds9, but you can see a graph of it by using a region
\end{itemize}
\end{enumerate}
\subsubsection{HR cube}
\label{sec:orgheadline10}
\begin{itemize}
\item Exactly the same, except that NV = 5614
\begin{itemize}
\item Wavelength scale: 0.85 angstroms
\item CRPIX3 = 1
\item CRVAL3 = 4595.
\end{itemize}
\item First try at dividing it up: do it by wavelength
\begin{itemize}
\item Divide into 8 parts of length 702
\begin{itemize}
\item Last one will be 700
\end{itemize}
\item Size will be 0.702 1.476 1.766 4 = 7.32 GB
\end{itemize}
\begin{center}
\begin{tabular}{rr}
Section & CRVAL3\\
\hline
0 & 4595.00\\
1 & 5191.70\\
2 & 5788.40\\
3 & 6385.10\\
4 & 6981.80\\
5 & 7578.50\\
6 & 8175.20\\
7 & 8771.90\\
\end{tabular}
\end{center}
\end{itemize}

\begin{verbatim}
from astropy.io import fits
from astropy import wcs
import numpy as np

hdulist = fits.open('DATA/DATACUBEFINALuser_20140216T010259_78380e1d.fits')
datcube = hdulist['DATA']
errcube = hdulist['STAT']
sections = np.arange(8, dtype=int)
NV = 702
k1_list = sections*NV
k2_list = k1_list + NV
wav0_list = datcube.header['CRVAL3'] + datcube.header['CD3_3']*NV*sections
for section, k1, k2, wav0 in zip(sections, k1_list, k2_list, wav0_list):
    fn = 'muse-hr-data-wavsec{}.fits'.format(section)
    hdr = datcube.header.copy()
    hdr['CRVAL3'] = wav0
    hdr['NAXIS3'] = NV
    print('Writing', fn)
    fits.PrimaryHDU(header=hdr, data=datcube.data[k1:k2]).writeto(fn)
\end{verbatim}
\subsubsection{}
\label{sec:orgheadline49}
\subsubsection{Heliocentric correction}
\label{sec:orgheadline51}
Again, these snippets need to be run on the CRyA server where the big data cubes are
\begin{enumerate}
\item Looking for keywords in the top-level header
\label{sec:orgheadline50}
\begin{verbatim}
hdr = hdulist[0].header
hdr.tofile('HRcube.hdr', sep='\n', padding=False)
\end{verbatim}

\begin{verbatim}
SIMPLE  =                    T / file does conform to FITS standard             
BITPIX  =                    8 / number of bits per data pixel                  
NAXIS   =                    0 / number of data axes                            
EXTEND  =                    T / FITS dataset may contain extensions            
COMMENT   FITS (Flexible Image Transport System) format is defined in 'Astronomy
COMMENT   and Astrophysics', volume 376, page 359; bibcode: 2001A&A...376..359H 
DATE    = '2014-11-13T08:54:24' / file creation date (YYYY-MM-DDThh:mm:ss UT)   
ORIGIN  = 'TEST    '           / European Southern Observatory                  
TELESCOP= 'ESO-VLT-U4'         / ESO <TEL>                                      
INSTRUME= 'MUSE    '           / Instrument used.                               
RA      =            83.780509 / [deg] 05:35:07.3 RA (J2000) pointing           
DEC     =             -5.39556 / [deg] -05:23:44.0 DEC (J2000) pointing         
EQUINOX =                2000. / Standard FK5                                   
RADECSYS= 'FK5     '           / Coordinate system                              
EXPTIME =                   5. / Integration time                               
MJD-OBS =       56704.04374097 / Obs start                                      
DATE-OBS= '2014-02-16T01:02:59.219' / Observing date                            
UTC     =                3770. / [s] 01:02:49.000 UTC                           
LST     =             21901.85 / [s] 06:05:01.850 LST                           
PI-COI  = 'UNKNOWN '           / PI-COI name.                                   
OBSERVER= 'UNKNOWN '           / Name of observer.                              
PIPEFILE= 'DATACUBE_FINAL.fits' / Filename of data product                      
BUNIT   = '10**(-20)*erg/s/cm**2/Angstrom'                                      
DATAMD5 = '69173383d3718d3ddb46e187f4cc2954' / MD5 checksum                     
OBJECT  = 'M42-lr  '           / Original target.                               
CHECKSUM= 'NcfSOcZPNcdPNcZP'   / HDU checksum updated 2014-11-12T22:17:16       
DATASUM = '0       '           / data unit checksum updated 2014-11-12T22:17:16 
HIERARCH ESO OBS AIRM =     5. / Req. max. airmass                              
HIERARCH ESO OBS AMBI FWHM = 2. / Req. max. seeing 
... ETC ...
\end{verbatim}

So, this does have the info that we need: RA, DEC, MJD-OBS in particular
\end{enumerate}
\end{document}
